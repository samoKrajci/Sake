\section*{Sake}\label{sake}

\textit{A project for programming class (year 1, term 2) at Charles University,
Prague.}
\medskip

Sake is an arcade-like multiplayer game, that uses monogame C\# library. Last snake slithering wins.

\subsection*{How to run the game}\label{how-to-run-the-game}

In order to run the game, server must be also running.

\subsubsection*{Server}\label{server}

To run server, you have to build and run \emph{server} project. There
are 3 commands in the \emph{server} program:

\begin{itemize}
\itemsep1pt\parskip0pt\parsep0pt
\item
  \textbf{lobby}: puts server in a lobby state, enabling users to
  connect. Max 4 players are supported.
\item
  \textbf{game}: starts the game
\item
  \textbf{quit}: terminates the program
\end{itemize}

\subsubsection*{Game}\label{game}

First you have to configure IP address of the server in the
\emph{constants} library in the \emph{Sake} project (use
\texttt{"localhost"} for localhost) To run the game, build and run
\emph{Sake} project. Then wait for the server to star the game.
\textbf{Do not terminate the program until the game ends, the server
will crash} (a feature, not a bug...).

Use right and left arrow to turn.

\medskip
\emph{It is not recommended to publish the game yet, any sort of
configuration is missing, so in order to change server IP address or
game paramateres, you have to change the code}
