\documentclass{article}
\usepackage[utf8]{inputenc}
\usepackage[T1]{fontenc}
\usepackage[a4paper, total={6.5in, 10in}]{geometry}
\setlength{\parindent}{0pt}
\usepackage{mathtools}
\usepackage{tcolorbox}
\DeclarePairedDelimiter\ceil{\lceil}{\rceil}
\DeclarePairedDelimiter\floor{\lfloor}{\rfloor}

\usepackage{makecell}

\renewcommand\theadalign{bc}
\renewcommand\theadfont{\bfseries}
\renewcommand\theadgape{\Gape[4pt]}
\renewcommand\cellgape{\Gape[4pt]}

\usepackage{listings}
\usepackage{color}
\definecolor{lightgray}{gray}{0.95}

\lstset{
    showstringspaces=false,
    basicstyle=\ttfamily,
    keywordstyle=\color{blue},
    commentstyle=\color[grey]{0.6},
    stringstyle=\color[RGB]{255,150,75}
}

\newcommand{\code}[1]{\colorbox{lightgray}{\lstinline{#1}}}

\title{Sake}
\author{Samuel Krajči}
\date{}

\begin{document}

\maketitle

\section{Úvod}

Základné pokyny pre užívateľa sa nachádzajú v \code{README.md} v git repozitári:

\bigskip

\begin{tcolorbox}
\section*{Sake}\label{sake}

\textit{A project for programming class (year 1, term 2) at Charles University,
Prague.}
\medskip

Sake is an arcade-like multiplayer game, that uses monogame C\# library. Last snake slithering wins.

\subsection*{How to run the game}\label{how-to-run-the-game}

In order to run the game, server must be also running.

\subsubsection*{Server}\label{server}

To run server, you have to build and run \emph{server} project. There
are 3 commands in the \emph{server} program:

\begin{itemize}
\itemsep1pt\parskip0pt\parsep0pt
\item
  \textbf{lobby}: puts server in a lobby state, enabling users to
  connect. Max 4 players are supported.
\item
  \textbf{game}: starts the game
\item
  \textbf{quit}: terminates the program
\end{itemize}

\subsubsection*{Game}\label{game}

First you have to configure IP address of the server in the
\emph{constants} library in the \emph{Sake} project (use
\texttt{"localhost"} for localhost) To run the game, build and run
\emph{Sake} project. Then wait for the server to star the game.
\textbf{Do not terminate the program until the game ends, the server
will crash} (a feature, not a bug...).

Use right and left arrow to turn.

\medskip
\emph{It is not recommended to publish the game yet, any sort of
configuration is missing, so in order to change server IP address or
game paramateres, you have to change the code}

\end{tcolorbox}

\pagebreak
\section{Viac o hre}

Hra je inšpirovaná klasickou hrou \textit{Snake} a online hrou \textit{Curve Fever}. Každý hráč má svojho hada ktorého vie ovládať, cieľom je ostať posledný v hre. Hady sa pohybujú po štvorčekovej sieti na tóruse (vedia prechádzať cez okraje), vedia sa pohybovať iba do štyroch smerov. Hra prebieha v \textit{kolách}, teda diskrétne, nie spojito, za sekundu sa štandardne udeje 10 kôl.V hre sa nachádzjú iba dva typy objektov. Hady a \textit{powerupy}. 

\subsection{Powerupy}
Hady vedia zbierať powerupy tak, že cez ne prejdú. Powerupy môžu dať hadu \textit{vlastnosť}, alebo vykonajú \textit{okamžitú akciu}. Powerupy sa objavujú v náhodných časoch a na náhodných miestach, každý s inou pravdepodobnosťou. V hre je 6 powerupov:

\begin{itemize}
    \item \textbf{food} \textit{(jablko)}: ako v klasickom \textit{snake}, had narastie o 1 políčko.
    \item \textbf{mega food} \textit{(hamburger)}: podoblné ako \textit{food}, ale had narastie o 5 políčok.
    \item \textbf{invincibility} \textit{(zlatá huba)}: spraví hada nesmrteľným na 30 kôl.
    \item \textbf{slow} \textit{(slimák)}: spraví hada dva krát pomalším, hýbe sa len každé druhé kolo po dobu 40 kôl.
    \item \textbf{reverse} \textit{(zrkadlo)}: obráti hada. Hlavu má tam, kde mal chvost a naopak. 
    \item \textbf{stone} \textit{(kameň)}: tento powerup sa nedá zobrať, je to iba prekážka, do ktorej ak had narazí, zomrie.
\end{itemize}

\subsection{Smrť}
Had vie zomrieť iba dvoma spôsobmi:
\begin{itemize}
    \item nabúraním do kameňa (\textit{stone}),
    \item alebo nabúraním do hociakého hada, vrátane seba.
\end{itemize}

Ak je had nesmrteľný (po vzatí powerupu \textit{invincibility}), nevie zomrieť ani jedným z týchto spôsobov.

\pagebreak
\section{Štruktúra programu}

Program pooužíva knižnicu \textit{MonoGame}.

Hra je zložená z dvoch projektov (\textit{server} a \textit{klient} (\textit{Sake})) a štyroch knižníc (\textit{constants}, \textit{game library}, \textit{protocol library} a \textit{client library}).

\subsection{Knižnice}

\subsubsection{Constants}

Tu nájdeme všetky konštanty ako napríklad \textit{rozmery mapy}, alebo \textit{trvania powerupov}.

\subsubsection{Game library}

Obsahuje viditeľné objekty v hre:

\paragraph{Snake}\mbox{} \\

Táto trieda predstavuje hada. Má všetky vlastnosti popisujúce hada ako \textit{dĺžka}, \textit{poloha}, \textit{smer} alebo jeho stav (ktorý je ovplyvňovaný \textit{powerupmi}). Taktiež má metódy, ktoré s ním hýbu, alebo ho vykresľujú.

\medskip
Má podtriedu \textit{SnakeUser}, ktorá obsahuje navyše iba nasledujúci smer hada.

\paragraph{Powerup}\mbox{} \\

Jednoduchá trieda, objekt triedy predstavuje powerup, ktorý má priradený \textit{typ}, \textit{polohu} a \textit{textúru}. Má iba metódu na vykreslenie.

\paragraph{Map}\mbox{} \\

Táto trieda popisuje celú hraciu plochu (resp. \textit{mapu}). Obsahuje \textit{rozmery mapy}, \textit{zoznam hadov} a \textit{zoznam powerupov}, a metódu \textit{updateFromMapUpdatePacket}, ktorá mení stav mapy, bližšie to popíšeme pri \textit{klient - server komunikácii}.

\medskip
Zložitejšia je ale podtrieda \textit{MasterMap}, ktorá má (pre efektívu detekciu kolízií) navyše 

\begin{itemize}
    \item (pre efektívu detekciu kolízií) uložený obsah každého políčka na mape (v slovníku),
    \item metódu na \textit{generovanie powerupov},
    \item metódu na \textit{vykreslenie}, celej mapy, zavolá iba vykreslenie všetkých objektov ktoré sa na nej nachádzajú,
    \item metódu \textit{AutoUpdate}, ktorá sa volá v každom kole a \textit{updatene stav celej mapy},
    \item a metódu \textit{createMapUpdatePacket}, ktorú bližšie popíšeme v \textit{klient - server komunikácii}.
\end{itemize}

Program \textit{klienta} používa popis stavu mapy triedu \textit{Map} a \textit{server} používa \textit{MasterMap}, viac v \textit{klient - server komunikácii}.

\paragraph{Rand}\mbox{} \\

Statická trieda na generovanie náhodných čísel.

\subsubsection{Client library}

Tu sa nachádza iba jedna trieda \textit{TcpClient}, ktorá obsahuje objekt striedy \textit{Socket} z knižnice \textit{System.Net.Sockets}. Objekt tejto triedy je v programe \textit{klienta} a zabezpečuje komunikáciu so serverom cez \textit{TCP protokol}. Má teda všetky potrebné metódy ako napríklad \textit{connectToServer}, \textit{disconnect}, \textit{requestLoop}, ...

\subsubsection{Protocol library}

Tu nájdeme štyri triedy paketov, ktoré sa v komunikácii používajú. Každá trieda má niekoľko parametrov a navyše má aj jeden parameter typu string (\textit{serialized}) v ktorom sú parametre objektu serializované a teda sú pripravené byť \textit{payload} v \textit{TCP pakete}. 

\begin{itemize}
    \item \textbf{PowerupInfo} - popisuje stav powerupu,
    \item \textbf{SnakeInfo} - popisuje stav hada,
    \item \textbf{MapInfo} - popisuje zmenu stavu mapy,
    \item \textbf{InitialInfo} popisuje počiatočný stav mapy.
\end{itemize}

Každá trieda má dva typy konštruktorov - \textit{zo serializovaného stringu} a \textit{klasický, explicitný} a teda vieme jednoducho informácie serializovať alebo rozbaliť.

\subsection{Klient}

Klient si sa skladá vpodstate iba z troch objektov - \textit{TcpClient}, \textit{Map} a \textit{SnakeUser}. O vykresľovanie sa stará kližnica \textit{MonoGame}. 

Na klientovi bežia dva procesy paralelne - \textit{Update}, ktorý vykresľuje mapu a paralelne \textit{RunTaskAfterResponseLoopAsync}, funkcia objektu \textit{TcpClient}, ktorá sa stará o komunikáciu so serverom.

\subsection{Server}

Server sa skladá vpodstate iba z dvoch objektov - \textit{MasterMap} ktorý popisuje stav celej hry a \textit{Socket}, objekt knižnice \textit{System.Net.Sockets}, ktorý odosiela pakety klientom.

\subsection{Klient - server komunikácia}

Komunikácia prebieha cez \textit{TCP protokol}. \textit{Server} má všetky inforácie o hre.
Keď je server v stave \textit{lobby}, klienti sa vedia pripojiť. Po prejdení do stavu \textit{game} server rozpošle paket popisujúci začiatočný stav hry (objekt triedy \textit{InitialInfo}). Potom začne nasledovný loop:

\begin{center}
    \begin{tabular}{|c|c|c|}
        \hline
        Fáza & \textbf{Server} & \textbf{Klient} \\
        \hline
        \hline
        1 & \makecell{pýta sa všetkých klientov na \\zmenu smeru (\code{"requesting move"})} & čaká na výzvu od serveru \\
        \hline
        2 & čaká na odpovede od všetkých klientov & \makecell{odpovedá serveru zmenou smeru\\ (\code{"l"}/\code{"r"}/\code{"f"})}\\
        \hline
        3 & \makecell{updatne mapu a pošle zmenu\\stavu (objekt \textit{MapInfo}, \\ktorý vráti metóda\\ \textit{createMapUpdatePacket} objektu \\\textit{MasterMap})} & čaká na update mapy \\
        \hline
        4 & čaká kým ubehne dĺžka jedného kola & \makecell{updatne stav metódou \\\textit{updateFromMapUpdatePacket} \\objektu \textit{Map}} \\
        \hline
    \end{tabular}
\end{center}

Server hru ukonči, keď ostanú menej ako dva hady nažive správou \code{"game over"} a následne ukonči všetky spojenia.

\end{document}
